\documentclass[12pt]{report}
\usepackage{../header}

\title{Тема 1. Дифференциальные уравнения.}

\begin{document}
	\maketitle
	
	\begin{tcolorbox}
		\textbf{УЧЕБНИК ДНЯ:}
		А.Ф. Филиппов. Сборник задач по дифференциальным уравнениям.
		\url{http://kvm.gubkin.ru/pub/uok/FilippovDU.pdf}
	\end{tcolorbox}
	
\section{Мотивация}
Зачем вообще мы изучаем решение дифференциальных уравнений?

Все темы этого курса строятся вокруг проблемы решения уравнения Лапласа. Вот это уравнение:
 \begin{equation}\label{eq:Laplace}
 \Delta\phi = 0.
 \end{equation}
  
\begin{rmk}
	 Это уравнение описывает распределение потенциала в области с заданными граничными условиями при отсутствии заряда в этой области. Если бы заряд был, уравнение Лапласа выглядело бы как
	\[
	\Delta\phi \equiv \nabla^2\phi = \rho,
	\]
	и называлось бы уравнением Пуассона. Учитывая, что напряжённость электрического поля $\vec E = -\nabla\phi$, уравнение Пуассона -- это одно из уравнений Максвелла.\qed
\end{rmk}

Уравнение~\eqref{eq:Laplace} решается (при благоприятном стечении обстоятельств\footnote{То есть, когда геометрия области позволяет это сделать.}) методом разделения переменных.	При этом возникает так называемая задача Штурма-Лиувилля (Ш-Л). Решение задачи Ш-Л -- это решение системы \textbf{дифференциальных уравнений}. Вот для этого и существует Тема 1.

\section{Основные понятия}
\emph{Неоднородным} дифференциальным уравнением порядка $n$ называется уравнение вида
\begin{equation}\label{eq:inhomogenous}
	F\left(x,y,y',y'',\dots, y^{(n)}\right) = g(x).
\end{equation}
В уравнении выше, $x$ -- независимая переменная, $y(x)$ -- искомая функция.

Если уравнение~\eqref{eq:inhomogenous} записывается в виде
\begin{equation}\label{eq:inhomogenous:linear}
a_0y + a_1y' + \dots + a_ny^{(n)} = g(x),
\end{equation}
то оно называется \emph{линейным}.\footnote{При этом, коэффициенты $a_0, \dots, a_n$ в общем случае могут быть функциями.}\\ 

Если $\forall x \Big[g(x)= 0\Big]$ уравнение~\eqref{eq:inhomogenous} становится \emph{однородным}:
\begin{equation}\label{eq:homogenous}
F\left(x,y,y',y'',\dots, y^{(n)}\right) = 0.
\end{equation}

\begin{rmk}[Формы записи]
	При записи дифференциального уравнения $F(x,y,y') = g(x)$ в форме
	\begin{equation*}
	\frac{\rd y}{\rd x} = f(x,y),
	\end{equation*}
	оно называется \emph{разрешённым} относительно производной; а при записи в форме
	\begin{equation*}
	M(x,y)\rd x + N(x,y)\rd y = f(x,y).
	\end{equation*}
	уравнением, записанным в \emph{полных дифференциалах}.
\end{rmk}

\begin{defn}[Задача Коши]
	Задачей Коши называется система
	\begin{equation}\label{eq:Cauchy}
		\begin{cases}
		y'(x) &= f(x,y), \\
		y(x_0) &= y_0;
		\end{cases}
	\end{equation}
	т.е. дифференциальное уравнение + начальные условия.\footnote{Во множественном, потому что количество начальных условий необходимых для решения уравнения порядка $n$ равно $n-1$: $y(x_0)$, $y'(x_0)$, $y''(x_0)$, ..., $y^{(n-1)}(x_0)$.} Решить задачу Коши значит отыскать решение дифференциального уравнения на данной области определения $X\ni x$, удовлетворяющее заданному начальному условию.
\end{defn}

\section{Что важно понимать о \emph{решении} дифференциального уравнения?}
\subsection{Различают \emph{общее} и \emph{частное} решения}
Частное решение -- это любая дифференцируемая на области определения функция $y = \phi(x)$, удовлетворяющая задаче Коши. Общее решение -- это семейство всех таких функций: $y = \phi(x, C)$, где $C$ -- символ \emph{произвольной} постоянной.
\begin{rmk}[Откуда взялась константа]
	В конечном итоге, чтобы найти решение $y$ дифференциального уравнения $y' = f(x,y)$, нужно проинтегрировать его производную: 
	\[
		y = \int \rd y = \int f(x,y)\rd x.
	\]
	Как известно, при интегрировании возникает константа, поскольку производная константы равна нулю, а значит для любого дифференциала верно: $\rd y = \rd(y + C)$. \qed
\end{rmk}
\subsection{Общее решение \emph{линейного неоднородного} уравнения -- это сумма...} 
... \emph{общего} решения соответствующего \emph{однородного} уравнения, и \emph{частного} решения \emph{неоднородного}:
\begin{equation}\label{eq:general-inhomogenous-soln}
	y_{\text{о.н.}} = y_{\text{о.о.}} + y_{\text{ч.н.}}.
\end{equation}

\section{Типы уравнений и методы их решения}
Мы рассмотрим следующие типы уравнений:
\begin{enumerate}
	\item с разделяющимися переменными;
	\item однородные;
	\item в полных дифференциалах.
\end{enumerate}

\subsection{Уравнения с разделяющимися переменными}
Это уравнения $y' = f(x,y)$, в которых правую часть можно представить произведением функций одной переменной:
\begin{equation*}
	y' = \frac{\rd y}{\rd x} = \phi(x)\psi(y).
\end{equation*}
Решаются элементарно: нужно перебросить всё, что с $x$ в одну сторону от равно, всё, что с $y$ -- в другую. Получим~\cite{Kyasov}
\[
	\frac{\rd y}{\psi(y)} = \phi(x)\rd x.
\]

Интегрируем; записываем ответ.

\textbf{Пример.} (Филиппов 51).
\[
xy\rd x + (x+1)\rd y = 0.
\]

Во-первых, поделим уравнение на $y(x+1)$:
\[
  \frac{x}{x+1}\rd x = -\frac{\rd y}{y}.
\]
Сделаем замену $t = x+1$; поскольку $\rd t = \rd (x+1) = \rd x$, и $\rd y/y = \rd\ln y$, получим:
\[
  \frac{t-1}{t}\rd t  = 1 \rd t - \rd\ln t = -\rd\ln y.
\]
Интегрируем:
\begin{align*}
\int\rd t - \int\rd\ln t &= -\int\rd\ln y, \\
t + C - \ln t &= - \ln y, \\
t + C &= \ln t - \ln y = \ln(t/y), 
\intertext{домножим обе части на -1:}
-t + \tilde{C} &= \ln (y/t), \\
y/t &= e^{-t +  \tilde{C}} = e^{-x -1 + \tilde{C}} =  \hat{C}e^{-x}, \\
y &= \hat{C}(x+1)e^{-x}. \qed
\end{align*}

\begin{thebibliography}{9}
	\bibitem{Kyasov}
	С.Н. Киясов, В.В. Шурыгин. Дифференциальные уравнения. Основы теории, методы решения задач.
	\url{https://kpfu.ru/docs/F931321200/kiyasov_shurygin.pdf}
	\bibitem{Ipatova}
	В.М. Ипатова, О.А. Пыркова, В.Н. Седов. Дифференциальные уравнения. Методы решений.
	\url{https://mipt.ru/education/chair/mathematics/upload/636/f_5ztibp-arphdx5wxdp.pdf}
\end{thebibliography}
\end{document}