\documentclass[12pt]{report}
\usepackage{../header}

\title{Тема 1. Дифференциальные уравнения.}

\begin{document}
	\maketitle
	
	\begin{tcolorbox}
		\textbf{УЧЕБНИК ДНЯ:}
		А.Ф. Филиппов. Сборник задач по дифференциальным уравнениям.
		\url{http://kvm.gubkin.ru/pub/uok/FilippovDU.pdf}
	\end{tcolorbox}
	
\section{Мотивация}
Зачем вообще мы изучаем решение дифференциальных уравнений?

Все темы этого курса строятся вокруг проблемы решения уравнения Лапласа. Вот это уравнение:
 \begin{equation}\label{eq:Laplace}
 \Delta\phi = 0.
 \end{equation}
  
\begin{rmk}
	 Это уравнение описывает распределение потенциала в области с заданными граничными условиями при отсутствии заряда в этой области. Если бы заряд был, уравнение Лапласа выглядело бы как
	\[
	\Delta\phi \equiv \nabla^2\phi = \rho,
	\]
	и называлось бы уравнением Пуассона. Учитывая, что напряжённость электрического поля $\vec E = -\nabla\phi$, уравнение Пуассона -- это одно из уравнений Максвелла.\qed
\end{rmk}

Уравнение~\eqref{eq:Laplace} решается (при благоприятном стечении обстоятельств\footnote{То есть, когда геометрия области позволяет это сделать.}) методом разделения переменных.	При этом возникает так называемая задача Штурма-Лиувилля (Ш-Л). Решение задачи Ш-Л -- это решение системы \textbf{дифференциальных уравнений}. Вот для этого и существует Тема 1.

\section{Основные понятия}
\emph{Неоднородным} дифференциальным уравнением порядка $n$ называется уравнение ``с правой частью,'' т.е. уравнение вида
\begin{equation}\label{eq:inhomogenous}
	F\left(x,y,y',y'',\dots, y^{(n)}\right) = g(x).
\end{equation}
В уравнении выше, $x$ -- независимая переменная, $y(x)$ -- искомая функция.

Если уравнение~\eqref{eq:inhomogenous} записывается в виде
\begin{equation}\label{eq:inhomogenous:linear}
a_0y + a_1y' + \dots + a_ny^{(n)} = g(x),
\end{equation}
то оно называется \emph{линейным}.\footnote{При этом, коэффициенты $a_0, \dots, a_n$ в общем случае могут быть функциями.}\\ 

Если $\forall x \Big[g(x)= 0\Big]$ уравнение~\eqref{eq:inhomogenous} становится \emph{однородным}:
\begin{equation}\label{eq:homogenous}
F\left(x,y,y',y'',\dots, y^{(n)}\right) = 0.
\end{equation}

\begin{rmk}[Формы записи]
	При записи дифференциального уравнения $F(x,y,y') = g(x)$ в форме
	\begin{equation*}
	\frac{\rd y}{\rd x} = f(x,y),
	\end{equation*}
	оно называется \emph{разрешённым} относительно производной; а при записи в форме
	\begin{equation}\label{eq:differential-form}
	M(x,y)\rd x + N(x,y)\rd y = f(x,y).
	\end{equation}
	уравнением, записанным в \emph{дифференциалах}.
\end{rmk}

\begin{defn}[Задача Коши]
	Задачей Коши называется система
	\begin{equation}\label{eq:Cauchy}
		\begin{cases}
		y'(x) &= f(x,y), \\
		y(x_0) &= y_0;
		\end{cases}
	\end{equation}
	т.е. дифференциальное уравнение + начальные условия.\footnote{Во множественном, потому что количество начальных условий необходимых для решения уравнения порядка $n$ равно $n-1$: $y(x_0)$, $y'(x_0)$, $y''(x_0)$, ..., $y^{(n-1)}(x_0)$.} Решить задачу Коши значит отыскать решение дифференциального уравнения на данной области определения $X\ni x$, удовлетворяющее заданному начальному условию.
\end{defn}

\section{Что важно понимать о \emph{решении} дифференциального уравнения?}
\subsection{Различают \emph{общее} и \emph{частное} решения}
Частное решение -- это любая дифференцируемая на области определения функция $y = \phi(x)$, удовлетворяющая задаче Коши. Общее решение -- это семейство всех таких функций: $y = \phi(x, C)$, где $C$ -- символ \emph{произвольной} постоянной.
\begin{rmk}[Откуда взялась константа]
	В конечном итоге, чтобы найти решение $y$ дифференциального уравнения $y' = f(x,y)$, нужно проинтегрировать его производную: 
	\[
		y = \int \rd y = \int f(x,y)\rd x.
	\]
	Как известно, при интегрировании возникает константа, поскольку производная константы равна нулю, а значит для любого дифференциала верно: $\rd y = \rd(y + C)$. \qed
\end{rmk}
\subsection{Общее решение \emph{линейного неоднородного} уравнения -- это сумма...} 
... \emph{общего} решения соответствующего \emph{однородного} уравнения, и \emph{частного} решения \emph{неоднородного}:
\begin{equation}\label{eq:general-inhomogenous-soln}
	y_{\text{о.н.}} = y_{\text{о.о.}} + y_{\text{ч.н.}}.
\end{equation}

\section{Типы уравнений и методы их решения~\cite{Kyasov}}
Мы рассмотрим следующие типы уравнений:
\begin{enumerate}
	\item с разделяющимися переменными;
	\item однородные;
	\item в полных дифференциалах.
\end{enumerate}

\subsection{Уравнения с разделяющимися переменными}
Это уравнения $y' = f(x,y)$, в которых правую часть можно представить произведением функций одной переменной:
\begin{equation*}
	y' = \frac{\rd y}{\rd x} = \phi(x)\psi(y).
\end{equation*}
\emph{Решаются} элементарно: нужно \emph{перебросить} всё, что с $x$ в одну сторону от равно, всё, что с $y$ -- в другую. Получим
\[
	\frac{\rd y}{\psi(y)} = \phi(x)\rd x.
\]

Интегрируем; записываем ответ.

\paragraph{Пример} (Филиппов 51).
\[
xy\rd x + (x+1)\rd y = 0.
\]

Во-первых, поделим уравнение на $y(x+1)$:\footnote{При этом возникает вероятность, что формула, которую мы получим после интегрирования, не будет включать решение $(x,y) = (-1, 0)$; его в таком случае надо будет указать отдельно.}
\[
  \frac{x}{x+1}\rd x = -\frac{\rd y}{y}.
\]
Сделаем замену $t = x+1$; поскольку $\rd t = \rd (x+1) = \rd x$, и $\rd y/y = \rd\ln y$, получим:
\[
  \frac{t-1}{t}\rd t  = 1 \rd t - \rd\ln t = -\rd\ln y.
\]
Интегрируем:
\begin{align*}
\int\rd t - \int\rd\ln t &= -\int\rd\ln y, \\
t + C - \ln t &= - \ln y, \\
t + C &= \ln t - \ln y = \ln(t/y), 
\intertext{домножим обе части на -1:}
-t + \tilde{C} &= \ln (y/t), \\
y/t &= e^{-t +  \tilde{C}} = e^{-x -1 + \tilde{C}} =  \hat{C}e^{-x}, \\
y &= \hat{C}(x+1)e^{-x}. \qed
\end{align*}

\subsection{Однородные уравнения}
Под однородными в данном случае понимаются уравнения $y' = f(x,y)$ у которых правая часть удовлетворяет \emph{условию однородности}:
\[
	f(tx,ty) = t^m f(x,y),
\]
где $m$ -- порядок однородности.

Чтобы \emph{решить} такое уравнение достаточно сделать \emph{замену} $y = xt$.

\begin{rmk}\label{rmk:homogenous-equation-form}
	Однородные уравнения также можно представить в виде $y' = f(y/x)$.
\end{rmk}

\paragraph{Пример} (Филиппов 108).
\[
	xy' = y - xe^{y/x}.
\]
Поделим всё на $x$ и увидим, что перед нами действительно однородное уравнение в форме из Замечания~\ref{rmk:homogenous-equation-form}:
\[
	y' = \frac yx - e^{\frac yx}.
\]
Делаем замену $t = y/x$; тогда $y' = t'x + t$ и уравнение переписывается как
\[
	t'x + t = t - e^t
\]
сокращаем лишние $t$ и получаем
\[
	\frac{\rd t}{\rd x}x = - e^t.
\]
В принципе, это уже уравнение с разделяющимися переменными. Единственное, что я предлагаю \emph{поменять местами зависимую и независимую переменные}; т.е. искать решение не в форме $y = y(x)$, а в форме $x = x(y)$. Поэтому, вместо уравнения $\frac{\rd t}{\rd x} = f(x,t)$ будем решать уравнение $\frac{\rd x}{\rd t} = f^{-1}(x,t)$:
\[
	\frac{\rd x}{x} = -e^{-t}\rd t.
\]
Интегрируем:
\begin{align*}
\int \rd\ln x &= \int \left(-e^{-t}\right)\rd t, \\
\ln Cx &= e^{-t} = e^{-y/x}, \\
-y/x &= \ln\ln Cx;
\end{align*}
записываем ответ:
\[
	y = -x\ln\ln Cx.
\]

\begin{rmk}[Немного практической философии]
	Я хочу обратить ваше внимание на то, что нет никаких причин, чтобы не поменять ролями независимую и зависимую переменные. Во-первых, это \emph{легитимный трюк} решения задачи; во-вторых -- у этого есть некоторое философское обоснование. 
	
	Дело в том, что есть мнение, что у природы нет причинно-следственных связей. (Скажем, так считал Юм.) В этом случае, независимая и зависимая переменные не являются двумя отдельными единицами бытия; вместо этого они всего лишь стороны одной и той же медали. Я процитирую в этом отношении Бертрана Рассела~\cite[Lecture V]{Russell}:
	\begin{quote}
		The traditional conception of cause and effect is one which modern science shows to be fundamentally erroneous, and requiring to be replaced by a quite different notion, that of LAWS OF CHANGE. ... Thus, if we are to take the cause as one event and the effect as another, both must be shortened indefinitely. The result is that we merely have, as the embodiment of our causal law, a certain direction of change at each moment. Hence we are brought to \emph{differential equations} as embodying causal laws.
	\end{quote}
	Это, разумеется, связано с математизацией природы.\footnote{По этому поводу, см. у Хайдеггера~\cite[стр.~413--414]{Heidegger}: ``The classical example for the historical development of a science and even for its ontological genesis, is the rise of \emph{mathematical physics}. What is decisive for its development does not lie in its rather high esteem for the observation of `facts', nor in its `application' of mathematics in determining the character of natural processes; it lies rather in the way in which Nature herself is matematically projected.''} Потеря причинно-следственной связи, на сколько я понимаю происходит именно здесь. \qed
\end{rmk}

\section{Уравнения в \emph{полных} дифференциалах}
Уравнением в полных дифференциалах называется уравнение формы~\eqref{eq:differential-form}, в котором левая часть может быть заисана как дивверенциал некоторой функции $F(x,y)$:
\begin{align}
	M(x,y)\rd x + N(x,y)\rd y &= \rd F(x,y) \notag\\
										  &= \frac{\partial F}{\partial x}\rd x + \frac{\partial F}{\partial y}\rd y.\label{eq:full-differential-form}
\end{align}

Есть \emph{критерий} по которому можно судить, существует ли такая функция $F(x,y)$:
\[
	\forall (x,y)\in D \Bigg[\frac{\partial M(x,y)}{\partial y} = \frac{\partial N(x,y)}{\partial x}\Bigg]
\]

\begin{rmk}[Откуда взялся этот критерий]
	Посмотрим снова на уравнение~\eqref{eq:full-differential-form}. В самой правой части мы видим частные производные $F$ по $x$ и по $y$. Как известно из курса матанализа, в смешанных производных порядок дифференцирования не имеет значения, так что 
	\[
		\frac{\partial^2 F}{\partial x\partial y} \equiv \frac{\partial M}{\partial y} = \frac{\partial N}{\partial x} \equiv \frac{\partial^2 F}{\partial y\partial x}. 
	\]
	\qed
\end{rmk}

Чтобы решить уравнение в полных дифференциалах -- то есть найти функцию $F(x,y)$, нужно просто последовательно интегрировать её производные $F_x'$, $F_y'$. 

Обратите внимание, что если изначально уравнение однородное, т.е. мы решаем $\rd F(x,y) = 0$, то
\emph{ответом} будет $\int \rd F(x,y) = \boxed{F(x,y) = C}$.

\paragraph{Пример} (Филиппов 186).
\[
	2xy\rd x + (x^2 - y^2)\rd y = 0
\]
Посмотрим на критерий:
\begin{align*}
	(2xy)'_y &= 2x, \\
	(x^2 - y^2)'_x &= 2x,
\end{align*}
значит перед нами действительно уравнение в частных производных.

Решаем. Путь 1: исходим из того, что $F_x' = 2xy$.
\begin{align*}
	F &= \int 2xy\rd x = x^2y + C(y), \\
	\Big(x^2y + C(y)\Big)'_y &= x^2 + C'(y) = x^2 - y^2;
\shortintertext{значит}
	C'(y) &= -y^2, \\
	C(y) &= - \int y^2\rd y = -\frac{y^3}{3} + C_1.
\shortintertext{В итоге}
	F(x,y) &= x^2y -\frac{y^3}{3} + C_1, 
\shortintertext{и ответ:}
	3x^2y - y^3 &= C.
\end{align*}

Путь 2: исходим из $F_y' = x^2 - y^2$.
\begin{align*}
	F &= \int (x^2 - y^2)\rd y = x^2y - \frac{y^3}{3} + C(x), \\
	\Big(x^2y-\frac{y^3}{3} + C(x)\Big)'_x &= 2xy; \\
	2xy + C'(x) &= 2xy, \\
	C'(x) &= 0 \Rightarrow C(x) = C_1.
\shortintertext{Следовательно, ответ выглядит как}
	F(x,y) &= x^2y - \frac{y^3}{3} + C_1 = C_2, \\
	3x^2y - y^3 &= C.
\end{align*}

\begin{thebibliography}{9}
	\bibitem{Kyasov}
	С.Н. Киясов, В.В. Шурыгин. \emph{Дифференциальные уравнения. Основы теории, методы решения задач.}
	\url{https://kpfu.ru/docs/F931321200/kiyasov_shurygin.pdf}
	\bibitem{Ipatova}
	В.М. Ипатова, О.А. Пыркова, В.Н. Седов. \emph{Дифференциальные уравнения. Методы решений.}
	\url{https://mipt.ru/education/chair/mathematics/upload/636/f_5ztibp-arphdx5wxdp.pdf}
	\bibitem{Russell}
	Bertrand Russell. \emph{The Analysis of Mind.}
	\url{https://www.gutenberg.org/files/2529/2529-h/2529-h.htm#link2H_4_0008}
	\bibitem{Heidegger}
	Martin Heidegger. \emph{Being and Time}. [Macquarrie and Robinson translation 1962 edition] \url{http://pdf-objects.com/files/Heidegger-Martin-Being-and-Time-trans.-Macquarrie-Robinson-Blackwell-1962.pdf}
\end{thebibliography}
\end{document}