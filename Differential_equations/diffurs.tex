\documentclass[12pt]{report}
\usepackage{../header}

\title{Тема 1. Дифференциальные уравнения.}

\begin{document}
	\maketitle
	
	\begin{tcolorbox}
		\textbf{УЧЕБНИК ДНЯ:}
		А.Ф. Филиппов. Сборник задач по дифференциальным уравнениям.
		\url{http://kvm.gubkin.ru/pub/uok/FilippovDU.pdf}
	\end{tcolorbox}
	
\section{Мотивация}
Зачем вообще мы изучаем решение дифференциальных уравнений?

Все темы этого курса строятся вокруг проблемы решения уравнения Лапласа. Вот это уравнение:
 \begin{equation}\label{eq:Laplace}
 \Delta\phi = 0.
 \end{equation}
  
\begin{rmk}
	 Это уравнение описывает распределение потенциала в области с заданными граничными условиями при отсутствии заряда в этой области. Если бы заряд был, уравнение Лапласа выглядело бы как
	\[
	\Delta\phi \equiv \nabla^2\phi = \rho,
	\]
	и называлось бы уравнением Пуассона. Учитывая, что напряжённость электрического поля $\vec E = -\nabla\phi$, уравнение Пуассона -- это одно из уравнений Максвелла.\qed
\end{rmk}

Уравнение~\eqref{eq:Laplace} решается (при благоприятном стечении обстоятельств\footnote{То есть, когда геометрия области позволяет это сделать.}) методом разделения переменных.	При этом возникает так называемая задача Штурма-Лиувилля (Ш-Л). Решение задачи Ш-Л -- это решение системы \textbf{дифференциальных уравнений}. Вот для этого и существует Тема 1.

\section{Основные понятия}
\emph{Неоднородным} дифференциальным уравнением порядка $n$ называется уравнение ``с правой частью,'' т.е. уравнение вида
\begin{equation}\label{eq:inhomogenous}
	F\left(x,y,y',y'',\dots, y^{(n)}\right) = g(x).
\end{equation}
В уравнении выше, $x$ -- независимая переменная, $y(x)$ -- искомая функция.

Если уравнение~\eqref{eq:inhomogenous} записывается в виде
\begin{equation}\label{eq:inhomogenous:linear}
a_0y + a_1y' + \dots + a_ny^{(n)} = g(x),
\end{equation}
то оно называется \emph{линейным}.\footnote{При этом, коэффициенты $a_0, \dots, a_n$ в общем случае могут быть функциями.}\\ 

Если $\forall x \Big[g(x)= 0\Big]$ уравнение~\eqref{eq:inhomogenous} становится \emph{однородным}:
\begin{equation}\label{eq:homogenous}
F\left(x,y,y',y'',\dots, y^{(n)}\right) = 0.
\end{equation}

\begin{rmk}[Формы записи]
	При записи дифференциального уравнения $F(x,y,y') = g(x)$ в форме
	\begin{equation*}
	\frac{\rd y}{\rd x} = f(x,y),
	\end{equation*}
	оно называется \emph{разрешённым} относительно производной; а при записи в форме
	\begin{equation}\label{eq:differential-form}
	M(x,y)\rd x + N(x,y)\rd y = f(x,y).
	\end{equation}
	уравнением, записанным в \emph{дифференциалах}.
\end{rmk}

\begin{defn}[Задача Коши]
	Задачей Коши называется система
	\begin{equation}\label{eq:Cauchy}
		\begin{cases}
		y'(x) &= f(x,y), \\
		y(x_0) &= y_0;
		\end{cases}
	\end{equation}
	т.е. дифференциальное уравнение + начальные условия.\footnote{Во множественном, потому что количество начальных условий необходимых для решения уравнения порядка $n$ равно $n-1$: $y(x_0)$, $y'(x_0)$, $y''(x_0)$, ..., $y^{(n-1)}(x_0)$.} Решить задачу Коши значит отыскать решение дифференциального уравнения на данной области определения $X\ni x$, удовлетворяющее заданному начальному условию.
\end{defn}

\section{Что важно понимать о \emph{решении} дифференциального уравнения?}
\subsection{Различают \emph{общее} и \emph{частное} решения}
Частное решение -- это любая дифференцируемая на области определения функция $y = \phi(x)$, удовлетворяющая задаче Коши. Общее решение -- это семейство всех таких функций: $y = \phi(x, C)$, где $C$ -- символ \emph{произвольной} постоянной.
\begin{rmk}[Откуда взялась константа]
	В конечном итоге, чтобы найти решение $y$ дифференциального уравнения $y' = f(x,y)$, нужно проинтегрировать его производную: 
	\[
		y = \int \rd y = \int f(x,y)\rd x.
	\]
	Как известно, при интегрировании возникает константа, поскольку производная константы равна нулю, а значит для любого дифференциала верно: $\rd y = \rd(y + C)$. \qed
\end{rmk}
\subsection{Общее решение \emph{линейного неоднородного} уравнения -- это сумма...} 
... \emph{общего} решения соответствующего \emph{однородного} уравнения, и \emph{частного} решения \emph{неоднородного}:
\begin{equation}\label{eq:general-inhomogenous-soln}
	y_{\text{о.н.}} = y_{\text{о.о.}} + y_{\text{ч.н.}}.
\end{equation}

\section{Типы уравнений и методы их решения~\cite{Kyasov}}
Мы рассмотрим следующие типы уравнений:
\begin{enumerate}
	\item с разделяющимися переменными;
	\item однородные;
	\item в полных дифференциалах.
\end{enumerate}

\subsection{Уравнения с разделяющимися переменными}
Это уравнения $y' = f(x,y)$, в которых правую часть можно представить произведением функций одной переменной:
\begin{equation*}
	y' = \frac{\rd y}{\rd x} = \phi(x)\psi(y).
\end{equation*}
\emph{Решаются} элементарно: нужно \emph{перебросить} всё, что с $x$ в одну сторону от равно, всё, что с $y$ -- в другую. Получим
\[
	\frac{\rd y}{\psi(y)} = \phi(x)\rd x.
\]

Интегрируем; записываем ответ.

\paragraph{Пример} (Филиппов 51).
\[
xy\rd x + (x+1)\rd y = 0.
\]

Во-первых, поделим уравнение на $y(x+1)$:\footnote{При этом возникает вероятность, что формула, которую мы получим после интегрирования, не будет включать решение $(x,y) = (-1, 0)$; его в таком случае надо будет указать отдельно.}
\[
  \frac{x}{x+1}\rd x = -\frac{\rd y}{y}.
\]
Сделаем замену $t = x+1$; поскольку $\rd t = \rd (x+1) = \rd x$, и $\rd y/y = \rd\ln y$, получим:
\[
  \frac{t-1}{t}\rd t  = 1 \rd t - \rd\ln t = -\rd\ln y.
\]
Интегрируем:
\begin{align*}
\int\rd t - \int\rd\ln t &= -\int\rd\ln y, \\
t + C - \ln t &= - \ln y, \\
t + C &= \ln t - \ln y = \ln(t/y), 
\intertext{домножим обе части на -1:}
-t + \tilde{C} &= \ln (y/t), \\
y/t &= e^{-t +  \tilde{C}} = e^{-x -1 + \tilde{C}} =  \hat{C}e^{-x}, \\
y &= \hat{C}(x+1)e^{-x}. \qed
\end{align*}

\subsection{Однородные уравнения}
Под однородными в данном случае понимаются уравнения $y' = f(x,y)$ у которых правая часть удовлетворяет \emph{условию однородности}:
\[
	f(tx,ty) = t^m f(x,y),
\]
где $m$ -- порядок однородности.

Чтобы \emph{решить} такое уравнение достаточно сделать \emph{замену} $y = xt$.

\begin{rmk}\label{rmk:homogenous-equation-form}
	Однородные уравнения также можно представить в виде $y' = f(y/x)$.
\end{rmk}

\paragraph{Пример} (Филиппов 108).
\[
	xy' = y - xe^{y/x}.
\]
Поделим всё на $x$ и увидим, что перед нами действительно однородное уравнение в форме из Замечания~\ref{rmk:homogenous-equation-form}:
\[
	y' = \frac yx - e^{\frac yx}.
\]
Делаем замену $t = y/x$; тогда $y' = t'x + t$ и уравнение переписывается как
\[
	t'x + t = t - e^t
\]
сокращаем лишние $t$ и получаем
\[
	\frac{\rd t}{\rd x}x = - e^t.
\]
В принципе, это уже уравнение с разделяющимися переменными. Единственное, что я предлагаю \emph{поменять местами зависимую и независимую переменные}; т.е. искать решение не в форме $y = y(x)$, а в форме $x = x(y)$. Поэтому, вместо уравнения $\frac{\rd t}{\rd x} = f(x,t)$ будем решать уравнение $\frac{\rd x}{\rd t} = f^{-1}(x,t)$:
\[
	\frac{\rd x}{x} = -e^{-t}\rd t.
\]
Интегрируем:
\begin{align*}
\int \rd\ln x &= \int \left(-e^{-t}\right)\rd t, \\
\ln Cx &= e^{-t} = e^{-y/x}, \\
-y/x &= \ln\ln Cx;
\end{align*}
записываем ответ:
\[
	y = -x\ln\ln Cx.
\]

\begin{rmk}[Немного практической философии]
	Я хочу обратить ваше внимание на то, что нет никаких причин, чтобы не поменять ролями независимую и зависимую переменные. Во-первых, это \emph{легитимный трюк} решения задачи; во-вторых -- у этого есть некоторое философское обоснование. 
	
	Дело в том, что есть мнение, что у природы нет причинно-следственных связей. (Скажем, так считал Юм.) В этом случае, независимая и зависимая переменные не являются двумя отдельными единицами бытия; вместо этого они всего лишь стороны одной и той же медали. Я процитирую в этом отношении Бертрана Рассела~\cite[Lecture V]{Russell}:
	\begin{quote}
		The traditional conception of cause and effect is one which modern science shows to be fundamentally erroneous, and requiring to be replaced by a quite different notion, that of LAWS OF CHANGE. ... Thus, if we are to take the cause as one event and the effect as another, both must be shortened indefinitely. The result is that we merely have, as the embodiment of our causal law, a certain direction of change at each moment. Hence we are brought to \emph{differential equations} as embodying causal laws.
	\end{quote}
	Это, разумеется, связано с математизацией природы.\footnote{По этому поводу, см. у Хайдеггера~\cite[стр.~413--414]{Heidegger}: ``The classical example for the historical development of a science and even for its ontological genesis, is the rise of \emph{mathematical physics}. What is decisive for its development does not lie in its rather high esteem for the observation of `facts', nor in its `application' of mathematics in determining the character of natural processes; it lies rather in the way in which Nature herself is matematically projected.''} Потеря причинно-следственной связи, на сколько я понимаю происходит именно здесь. \qed
\end{rmk}

\section{Уравнения в \emph{полных} дифференциалах}
Уравнением в полных дифференциалах называется уравнение формы~\eqref{eq:differential-form}, в котором левая часть может быть заисана как дифференциал некоторой функции $F(x,y)$:
\begin{align}
	M(x,y)\rd x + N(x,y)\rd y &= \rd F(x,y) \notag\\
										  &= \frac{\partial F}{\partial x}\rd x + \frac{\partial F}{\partial y}\rd y.\label{eq:full-differential-form}
\end{align}

Есть \emph{критерий} по которому можно судить, существует ли такая функция $F(x,y)$:
\[
	\forall (x,y)\in D \Bigg[\frac{\partial M(x,y)}{\partial y} = \frac{\partial N(x,y)}{\partial x}\Bigg]
\]

\begin{rmk}[Откуда взялся этот критерий]
	Посмотрим снова на уравнение~\eqref{eq:full-differential-form}. В самой правой части мы видим частные производные $F$ по $x$ и по $y$. Как известно из курса матанализа, в смешанных производных порядок дифференцирования не имеет значения, так что 
	\[
		\frac{\partial^2 F}{\partial x\partial y} \equiv \frac{\partial M}{\partial y} = \frac{\partial N}{\partial x} \equiv \frac{\partial^2 F}{\partial y\partial x}. 
	\]
	\qed
\end{rmk}

Чтобы решить уравнение в полных дифференциалах -- то есть найти функцию $F(x,y)$, нужно просто последовательно интегрировать её производные $F_x'$, $F_y'$. 

Обратите внимание, что если изначально уравнение однородное, т.е. мы решаем $\rd F(x,y) = 0$, то
\emph{ответом} будет $\int \rd F(x,y) = \boxed{F(x,y) = C}$.

\paragraph{Пример} (Филиппов 186).
\[
	2xy\rd x + (x^2 - y^2)\rd y = 0
\]
Посмотрим на критерий:
\begin{align*}
	(2xy)'_y &= 2x, \\
	(x^2 - y^2)'_x &= 2x,
\end{align*}
значит перед нами действительно уравнение в частных производных.

Решаем. Путь 1: исходим из того, что $F_x' = 2xy$.
\begin{align*}
	F &= \int 2xy\rd x = x^2y + C(y), \\
	\Big(x^2y + C(y)\Big)'_y &= x^2 + C'(y) = x^2 - y^2;
\shortintertext{значит}
	C'(y) &= -y^2, \\
	C(y) &= - \int y^2\rd y = -\frac{y^3}{3} + C_1.
\shortintertext{В итоге}
	F(x,y) &= x^2y -\frac{y^3}{3} + C_1, 
\shortintertext{и ответ:}
	3x^2y - y^3 &= C.
\end{align*}

Путь 2: исходим из $F_y' = x^2 - y^2$.
\begin{align*}
	F &= \int (x^2 - y^2)\rd y = x^2y - \frac{y^3}{3} + C(x), \\
	\Big(x^2y-\frac{y^3}{3} + C(x)\Big)'_x &= 2xy; \\
	2xy + C'(x) &= 2xy, \\
	C'(x) &= 0 \Rightarrow C(x) = C_1.
\shortintertext{Следовательно, ответ выглядит как}
	F(x,y) &= x^2y - \frac{y^3}{3} + C_1 = C_2, \\
	3x^2y - y^3 &= C.
\end{align*}

\section{Фундаментальная система решений. Определитель Вронского}
\tldr{Любое решение линейного однородного дифференциального уравнения можно записать как линейную комбинацию ``базисных'' функций (ФСР, определение~\ref{def:FSS}). Проверить функции на линейную независимость можно вычислив их вронскиан (определение~\ref{def:Wronskian}). \textit{Найти} ФСР уравнения можно решив его характеристическое уравнение (раздел~\ref{sec:characteristic-equation}).}


Мы с вами изучаем \emph{линейные} дифференциальные уравнения. Решение линейного \emph{неоднородного} уравнения, в соответствии с уравнением~\eqref{eq:general-inhomogenous-soln}, сводится к двум фазам:
\begin{enumerate}[(1)]
	\item отыскание общего решения соответствующего однородного уравнения (решая характеристическое уравнение, см. раздел~\ref{sec:characteristic-equation}),\label{itm:phase1} и
	\item отыскание частного решения (методом вариации постоянной, раздел~\ref{sec:Lagrange-method}) самог\'{o} неоднородного уравнения.\label{itm:phase2}
\end{enumerate}

Рассмотрим первую фазу. Линейное однородное уравнение 
\[
a_ny^{(n)} + a_{n-1}y^{(n-1)} + \cdots + a_0y^{(0)} = 0
\]
можно формально записать как $Ly(x) = 0$, где $L$ -- линейный оператор дифференцирования:
\[
L = \sum_{k=1}^n a_k\frac{\rd^k}{\rd x^k},
\]
аналогичный оператору набла $\nabla = \hat x\frac{\partial}{\partial x} + \hat y\frac{\partial}{\partial y} + \hat z\frac{\partial}{\partial z}$.

Поскольку оператор $L$ линейный, он обладает следующими свойствами:
\begin{itemize}
	\item $L\left[y_1(x) + y_2(x)\right] = Ly_1(x) + Ly_2(x)$;
	\item $L\left[C\cdot y(x)\right] = C\cdot Ly(x)$, где $C = \const$.
\end{itemize}

Что в свою очередь означает, что для любой совокупности функций $y_1(x), y_2(x), \dots, y_m(x)$, функция $\phi(x) = \sum_k b_ky_k(x)$ -- тоже решение:\footnote{Напомню, что (частным) решением дифференциального уравнения $Ly(x)=0$ является функция $\psi(x)$, обращающая его в верное тождество: $L\psi(x) = 0$ -- истинно.}
\[
L\phi(x) = b_1\underbrace{Ly_1(x)}_{=0} + b_2\underbrace{Ly_2(x)}_{=0} + \cdots + b_m\underbrace{Ly_m(x)}_{=0} = 0.
\]

\begin{defn}[Фундаментальная система решений]\label{def:FSS}	
	Таким образом, решения линейного однородного дифференциального уравнения ${Ly(x) = 0}$  образуют ``линейное пространство''; то есть существует \emph{совокупность решений} $\phi_1(x), \dots, \phi_n(x)$, такая, что \emph{любое} решение $y(x)$ уравнения можно представить в форме их \emph{линейной комбинации}:
	\[
	\forall y(x) \left[y(x) = \sum_{k=1}^{n} C_k\phi_k(x) \rightarrow Ly(x) = 0\right]. 
	\]\qed
\end{defn}

Решения $\phi_1(x), \dots, \phi_n(x)$ имеют следующие свойства:
\begin{itemize}
	\item их \emph{количество}, $n$, равно \emph{порядку дифференцирования} при старшей производной в $L$;
	\item они -- как можно было предполагать -- \emph{линейно независимы} на области определения дифференциального уравнения.
\end{itemize}

\begin{defn}[Линейная независимость функций]
	Функции $\phi_1(x), \dots, \phi_n(x)$ называются \emph{линейно независимыми} на промежутке $[x_0,x_1]$ тогда, и только тогда, когда их \emph{линейная комбинация} ${a_1\phi_1(x) + \cdots + a_n\phi_n(x) = 0}$ \emph{тождественно} равна нулю на этом промежутке \emph{лишь в случае} ${a_1=a_2=\cdots=a_n=0}$:
	\[
		\forall x\in[x_0,x_1]\sum_{k=1}^n a_k\phi_k(x) = 0 \rightarrow \forall k~a_k=0.
	\] \qed
\end{defn}
\begin{defn}[Определитель Вронского]\label{def:Wronskian}
	Это функция $W(x| f_1, f_2, \dots, f_n)$, \emph{позволяющая проверить} функции $f_1(x), \dots, f_n(x)$ на \emph{линейную зависимость}.
	\[
		W(x|f_1,\dots,f_n) = 
			\begin{vmatrix}
				f_1(x)         & f_2(x)         & \cdots & f_n(x)         \\
				f_1'(x)        & f_2'(x)        & \cdots & f_n'(x)        \\
				\cdots         & \cdots         & \cdots & \cdots         \\
				f_1^{(n-1)}(x) & f_2^{(n-1)}(x) & \cdots & f_n^{(n-1)}(x)
			\end{vmatrix}.
	\]
	Если $\forall x\in[x_0, x_1] W(x| f_1, \dots, f_n) = 0$, то $f_1(x), \dots, f_n(x)$ -- линейно \emph{зависимые} функции. Если же $\exists \bar x\in [x_0, x_1]$ такая, что $W(\bar x| f_1,\dots,f_n) \neq 0$, то функции линейно \emph{независимы}.\footnote{У вронскиана есть интересное свойство, что \emph{в случае, если} $f_1,\dots,f_n$ -- \emph{решения} $Ly(x)=0$, вронскиан либо \emph{тождественно равен}, либо \emph{тождественно неравен}, нулю.} \qed
\end{defn}
\begin{defn}[Формула Остроградского-Лиувилля]
	Пусть в общем виде наше линейное дифференциальное уравнение выглядит как 
	\[
		a_n(x)y^{(n)}(x) + a_{n-1}y^{(n-1)}(x) + \cdots + a_0y = 0.
	\]
	Вронскиан ФСР этого уравнения можно записать как
	\[
		W(x) = W(x_0)\exp\left(-\int_{x_0}^{x} \frac{a_{n-1}(\xi)}{a_n(\xi)} \rd\xi\right),
	\]
	где $x_0$ -- некоторая точка области определения уравнения.
\end{defn}


\subsection{Характеристическое уравнение}\label{sec:characteristic-equation}

Характеристическим уравнением линейного однородного дифференциального уравнения с постоянными коэффициентами
\begin{align*}
	a_ny^{(n)}(x)+ a_{n-1}y^{(n-1)}(x)\cdots + a_0y(x) &= \\
	a_nD^ny(x) + a_{n-1}D^{n-1}y(x) + \cdots + a_0D^0y(x) &= \\
	\Big[a_nD^n + a_{n-1}D^{n-1} + \cdots + a_0D^0\Big]y(x) &\equiv \\
	L[D]y(x) &=0
\end{align*}
называется многочлен
\[
	L[\lambda] = a_n\lambda^n + a_{n-1}\lambda^{n-1}+\cdots+a_0\lambda = 0.
\]

\begin{rmk}[Откуда оно взялось]
	Вернёмся к определению~\ref{def:FSS} ФСР дифференциального уравнения. Два факта:
	\begin{enumerate}[(1)]
		\item ФСР -- это система \emph{базисных} векторов линейного пространства решений дифференциального уравнения;
		\item линейное дифференциальное уравнение $y^{(n)} = F(x, y, y', \dots, y^{(n-1)})$ порядка $n$ можно представить в виде \emph{системы} $n$ линейных дифференциальных уравнений первого порядка, путём замены: $y\mapsto y_1$, $y' = y_1' \mapsto y_2$, $y'' = (y_1')' = y_2' \mapsto y_3$ \&c, получая
		\[
			\begin{pmatrix}
			y_1' \\ y_2'\\ y_3'\\ \dots \\ y_n'
			\end{pmatrix} = 
			A\begin{pmatrix}
			y_2\\ y_3 \\ \dots \\ F(x, y_1, y_2, \dots, y_n)
			\end{pmatrix}.
		\]
	\end{enumerate}
	У матрицы $A$ (представляющей линейное отображение из пространства функций в пространство функций), разумеется есть \emph{собственные векторы} -- и они как раз образуют ФСР. \qed
\end{rmk}
\paragraph{Пример} Записать ФСР уравнения $\boxed{y'' - 6y' + 5y = 0}$.
Разрешим уравнение относительно старшей производной:
\[
	y'' = 6y' - 5y,
\]
и сделаем замену: $y \mapsto y_1$, $y' \mapsto y_2$, $y'' = y_2'$. Получим:
\[
	\begin{cases}
	y_1' &= y_2, \\
	y_2' &= -5y_1 + 6y_2;
	\end{cases}
\]
или другими словами: 
\[
	\vec y' = \begin{pmatrix}
	0 & 1 \\
	-5 & 6
	\end{pmatrix} \vec y.
\]
Найдём собственные числа матрицы; для этого составим \emph{характеристическое уравнение}: 
\begin{align*}
	|A - \lambda I| &= \begin{vmatrix}
	-\lambda & 1 \\
	-5 & 6-\lambda
	\end{vmatrix} = \lambda(\lambda-6) + 5 \\
	&= \boxed {\lambda^2 - 6\lambda + 5 = 0}
\end{align*}
Решения характеристического уравнения -- $\lambda_1 = 1$, $\lambda_2 = 5$. 

Что такое собственный вектор? Это такой вектор, для которого верно $A\vec u = \lambda \cdot \vec u$. Наши векторы -- решения дифференциального уравнения. У какой функции $u$ производная $u' = \lambda \cdot u$? У $u(x) = \exp(\lambda\cdot x)$. Поэтому собственные решения ищутся (в своём \emph{базовом} исполнении) именно в этой форме. Почему я сказал в базовой? Потому что корни характеристического уравнения могут оказаться кратными, и в этом случае собственное решение ищется в форме $u(x) = P_k(x)\cdot\exp(\lambda x)$, где $P_k(x) = \sum_{i=0}^{k-1}C_ix^i$. Случай, когда корень $\lambda = \alpha + i\beta \in \mathbb C$, аналогичен, просто экспоненту надо разложить по формуле Эйлера:
\[
	e^{(\alpha + i\beta x)} = e^{\alpha x}\cdot e^{i\beta x} = e^{\alpha x}\cdot\left(\cos\beta x + i\sin\beta x\right),
\]
(и затем убрать $i$ в знак константы).

Соответственно, ФСР выглядит как 
\[
	\lbrace e^{x}, e^{5x}\rbrace,
\]
а любое решение уравнения можно записать в виде
\[
	y(x) = C_1e^x + C_2e^{5x}.
\]

\section{Метод вариации постоянных}\label{sec:Lagrange-method}
Мы практически готовы решать неоднородные уравнения. Решая характеристическое уравнение (раздел~\ref{sec:characteristic-equation}) мы совершаем фазу~\ref{itm:phase1} решения. Теперь осталось научиться совершать фазу~\ref{itm:phase2} -- отыскивать частное решение неоднородного уравнения. 

Вообще говоря, его можно угадать (на основании формы правой части). Можно также использовать метод Бернулли -- представить искомую функцию как произведение двух. Но в этом разделе мы рассмотрим ещё один метод -- метод Лагранжа (метод вариации произвольной постоянной).

\emph{Суть} метода -- искать частное решение неоднородного уравнения в форме общего решения однородного.

\subsection{Уравнения первого порядка}
Для простоты, пусть у нас ДУ первого порядка: $y' + p(x)y = q(x)$, и мы нашли решение однородного уравнения в виде $y(x) = C\phi(x)$. А пусть теперь $C$ -- \emph{не константа, а некоторая функция}: $y(x) = u(x)\phi(x)$. Подставим в изначальное (неоднородное) уравнение:
\[
	u'\phi + u\phi' + pu\phi = q.
\]
Поскольку решение $y' + p(x)y = 0$ можно записать в форме ${\phi(x) = e^{-\int p(x)\rd x}}$ (это уравнение с разделяющимися переменными; можете проверить),  $\phi' = -p\phi$, и значит два члена в предыдущем уравнении сокращаются:
\begin{align*}
	u'\phi - \cancel{up\phi} + \cancel{pu\phi} &= q, \\
	u'\phi &= q, \\
	\frac{\rd u}{\rd x} &= q\phi^{-1} = qe^{+\int p(x)\rd x}.
\shortintertext{Соответственно}
	\int\rd u(x) &= \int q(x)e^{p(x)\rd x}\rd x + C.
\shortintertext{Возвращаемся к $y(x) = u(x)\phi(x)$:}
	y(x) &= \left[\int q(x)e^{p(x)\rd x}\rd x + C\right]\cdot e^{-\int p(x)\rd x}.
\end{align*}

\subsection{Уравнения высших порядков}
Теперь рассмотрим уравнение $a_ny^{(n)} + a_{n-1}y^{(n-1)} + \cdots + a_0y = q(x)$. 
В этом случае, заменив константы на функции в общем решении однородного уравнения
\[
	y(x) = u_1(x)\phi_1(x) + u_2(x)\phi_2(x) + \dots + u_n(x)\phi_n(x),
\]
мы составляем систему уравнений
\[
	\begin{cases}
	u_1'\phi_1 + u_2'\phi_2 + \dots + u_n'\phi_n &= 0, \\
	u_1'\phi_1' + u_2'\phi_2' + \dots + u_n'\phi_n' &= 0, \\
	\dots &= 0, \\
	u_1'\phi_1^{(n-1)} + u_2'\phi_2^{(n-1)} + \dots + u_n'\phi_n^{(n-1)} &= \sfrac{q(x)}{a_n}, \\
	\end{cases}
\]
которую нужно решить относительно $\vec u'$:
\[
\underbrace{\begin{pmatrix}
\phi_1 & \phi_2 & \cdots & \phi_n \\
\phi_1' & \phi_2'& \cdots &\phi_n' \\
\cdots & \cdots& \cdots & \cdots \\
\phi_1^{(n-1)}& \phi_2^{(n-1)} & \cdots & \phi_n^{(n-1)}
\end{pmatrix}}_{\text{определитель этой матрицы -- вронскиан (опр.~\ref{def:Wronskian})}} \vec u' = 
\begin{pmatrix}
0 \\ 0 \\ \cdots \\ q(x)/a_n
\end{pmatrix}
\]
Поскольку $\phi_1,\cdots,\phi_n$ линейно независимы, вронскиан матрицы слева отличен от нуля на всей области определения дифференциального уравнения, и значит существует уникальное решение.

\begin{thebibliography}{9}
	\bibitem{Tikhonov}
	А.Н. Тихонов, А.Б. Васильева, А.Г. Свешников. \emph{Дифференциальные уравнения.}
	\url{https://obuchalka.org/2015031483302/differencialnie-uravneniya-tihonov-a-n-vasileva-a-b-sveshnikov-a-g-2005.html}
	\bibitem{Kyasov}
	С.Н. Киясов, В.В. Шурыгин. \emph{Дифференциальные уравнения. Основы теории, методы решения задач.}
	\url{https://kpfu.ru/docs/F931321200/kiyasov_shurygin.pdf}
	\bibitem{Ipatova}
	В.М. Ипатова, О.А. Пыркова, В.Н. Седов. \emph{Дифференциальные уравнения. Методы решений.}
	\url{https://mipt.ru/education/chair/mathematics/upload/636/f_5ztibp-arphdx5wxdp.pdf}
	\bibitem{Russell}
	Bertrand Russell. \emph{The Analysis of Mind.}
	\url{https://www.gutenberg.org/files/2529/2529-h/2529-h.htm#link2H_4_0008}
	\bibitem{Heidegger}
	Martin Heidegger. \emph{Being and Time}. [Macquarrie and Robinson translation 1962 edition] \url{http://pdf-objects.com/files/Heidegger-Martin-Being-and-Time-trans.-Macquarrie-Robinson-Blackwell-1962.pdf}
\end{thebibliography}
\end{document}