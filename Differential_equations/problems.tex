\documentclass[12pt]{report}
\usepackage{../header}

\begin{document}
\paragraph{Филиппов 52}
\[
	\sqrt{y^2 + 1}\rd x = xy\rd y.
\]	
Решаем:
\begin{align*}
	\frac{\rd x}{x} &= \frac{y\rd y}{\sqrt{y^2+1}}, \\
	\rd\ln x &= \frac12 \frac{\rd y^2}{\sqrt{y^2+1}}, ~~ t = y^2 + 1;
\shortintertext{воспользуемся тем, что $\rd u = \rd (u + \const)$:}
	\rd \ln x = \frac12 \frac{\rd t}{\sqrt t} = \rd \sqrt t, \\
	\ln C|x| = \sqrt t = \sqrt{y^2 + 1}.
\end{align*}

\paragraph{Филиппов 53}
\begin{equation}\label{eq:Cauchy}
	\begin{cases}
	(x^2 - 1)y' + 2xy^2 &= 0, \\
	y(0) &= 1.
	\end{cases}
\end{equation}
Решение:
\begin{align*}
	2xy^2 &= (1-x^2)y', \\
	\frac{2x}{1-x^2}\rd x &= \frac{\rd y}{y^2}, \\
	\frac{\rd (x^2 - 1)}{1 - x^2} &= -\rd\left(\frac1y\right),
\shortintertext{заменим $t = x^2-1$}
	\frac{\rd t}{t} &= \rd\ln t = \rd y^{-1}, \\
	\ln C|x^2-1| = y^-1, \\
\shortintertext{следовательно \emph{общее} решение:}
	y\ln C|x^2-1| &= 1, ~~y = 0 ~~ \text{(потерянное при делении решение)}.
\end{align*}
Ищем частное решение. Для этого удобнее всего вынести константу из под знака логарифма:
\[
y\cdot \left(\ln|x^2-1| + \underbrace{\ln C}_{C_1}\right) = 1.
\]
\begin{align*}
	y(0)\cdot\left(\underbrace{\ln|0^2-1|}_{\ln 1 = 0} + C_1\right) &= 1, \\
	C_1 &= 1. 
\end{align*}
Итого, \emph{частное} решение, удовлетворяющее задаче Коши~\eqref{eq:Cauchy}:
\[
 	y\cdot\left(\ln|x^2| + 1\right) = 1.
\]

\paragraph{Филиппов 101}
\[
	(x+2y)\rd x - x\rd y = 0.
\]
Решение:
\begin{align*}
	y' &= 1 + 2\cdot \frac yx = 1 + 2t, \\
	y' &= t'x + t, \\
	t'x + t &= 1 + 2t, \\
	\frac{\rd t}{\rd x}x &= 1 + t, \\
	\frac{\rd x}{x} &= \frac{\rd(t+1)}{t+1}, \\
	\rd\ln x &= \rd\ln(t+1), \\
	C_1 + \ln x &= \ln(1 + \frac yx) = \ln(x + y) - \ln x, \tag{$\ln\frac ab = \ln a - \ln b$} \\
	\ln(x+y) &= \ln x^2 + \underbrace{\ln C}_{C_1} = \ln Cx^2, \tag{$a\cdot \ln x = \ln x^a$} \\
	x+y &= Cx^2,~~ x=0.
\end{align*}

\paragraph{Филиппов 102}
\[
	(x-y)\rd x + (x+y)\rd y = 0.
\]
Решение:
\begin{align*}
	y' &= \frac{y-x}{y+x} = \frac{t-1}{t+1} = t'x + t, \\
	\frac{\rd t}{\rd x}x &= -\frac{1+t^2}{1+t}, \\
	\frac{1+t}{1+t^2}\rd t &= -\rd\ln x, \\\\
	\underbrace{\frac{\rd t}{1+t^2}}_{\rd\arctan t} + \frac 12\underbrace{\frac{\rd(t^2+1)}{t^2+1}}_{\frac12\rd\ln(1+t^2)} &= -\rd\ln x, \\
	\arctan\frac yx + \frac12\ln(1 + \frac{y^2}{x^2}) + \underline{\ln x} &= \tilde{C},
\intertext{Домножим левую и правую части уравнения \underline{на 2} и заменим $\ln\frac{x^2 + y^2}{x^2} = \ln(x^2+y^2) - \ln x^2$}
	2\arctan\frac yx + \underbrace{\ln(x^2 + y^2) - \ln x^2 } _{\ln\frac{x^2+y^2}{x^2}}+ \underline{\ln x^2} &= C.
\end{align*}
Ответ:
\[
	\ln(x^2 + y^2) = C - 2\arctan\frac yx.
\]

\paragraph{Филиппов 189}
\[
	\frac yx\rd x + (y^3 + \ln x)\rd y = 0.
\]
Видим, что 
\[
	\begin{cases}
	F_y' &= y^3 + \ln x, \\
	F_x' &= y/x.
	\end{cases}
\]
Решаем:
\begin{align*}
	F &= \int(y^3 + \ln x)\rd y = y^4/4 + y\ln x + C(x).
\shortintertext{Возьмём производную вышестоящего выражения и приравняем её к $F_x'$:}
	y/x + C'(x) &= F_x' = y/x;
	C'(x) &= 0 \Rightarrow C = C_1.
\shortintertext{Соответственно:}
	F(x,y) &= y^4/4 + y\ln x + C_1;
\shortintertext{решением является}
	F(x,y) = C, \\
	y^4/4 + y\ln x &= C - C_1 \equiv C,
\end{align*}
и домножив всё на 4 для красоты получим ответ:
\[
	y^4 + 4y\ln x = C.
\]

\paragraph{Филиппов 511}
Решить 
\[
y'' + y' - 2y = 0.
\]
Составляем характеристическое уравнение:
\[
\lambda^2 + \lambda - 2 = 0,
\]
с корнями $\lambda_1 = 1$, $\lambda_2=-2$, и записываем ответ:
\[
y(x) = C_1e^x + C_2e^{-2x}.
\]

\paragraph{Филиппов 534}
\begin{equation}\label{eq:initial-534}
y'' + y = 4xe^x.
\end{equation}
Фаза 1: решаем соответствующее однородное уравнение.
Характеристическое уравнение $\lambda^2+1=0$ имеет сопряжённые комплексные корни $\lambda_{1,2} = \pm i$, и значит ответ запишется в виде
\[
	y(x) = C_1\cos x + C_2\sin x.
\]

Фаза 2: ищем частное решение неоднородного уравнения. Поскольку правая часть имеет форму $P_m(x)e^{\gamma x}$, решение можно найти \emph{методом неопределённых коэффициентов} (см.~\cite[стр.~50, ур-е (4)]{Filippov}). Частное решение ищем в виде 
\begin{align*}
	y_1(x) &= (A+Bx)e^x, \\
	y_1' &= (A+B)e^x + Bxe^x, \\
	y_1'' &= (A+2B)e^x + Bxe^x.
\end{align*}
Подставляем в уравнение~\eqref{eq:initial-534}:
\begin{align*}
	\underbrace{(A+2B)e^x + Bxe^x}_{y''} + \underbrace{Ae^x + Bxe^x}_{y} &= 4xe^x, \\
	2(A+B)e^x + 2Bxe^x &= 4xe^x.
\end{align*}
Приравнивая коэффициенты при подобных членах, получим:
\[
	\begin{cases}
	A &= -B, \\
	B &= 2.
	\end{cases}
\]
Частное решение $y_1(x) = (2x-2)e^x$, и полный ответ: 
\[
	y(x) = C_1\cos x + C_2\sin x + (2x-2)e^x.
\]

\paragraph{Филиппов 539}
РЕШИТЬ.

\paragraph{Филиппов 589}
Решить уравнение Эйлера
\[
	x^2y'' - 4xy + 6y = 0.
\]
Такие уравнения решаются заменой $x = e^t$; соответственно:
\begin{align*}
	\rd x &= e^t\rd t, \\
	\frac{\rd }{\rd x} y &= \frac{\rd y}{\rd t}\cdot \frac{\rd t}{\rd x} = \dot y e^{-t}, \\
	\frac{\rd}{\rd x} \left(\dot y e^{-t}\right) &= \left[\frac{\rd \dot y}{\rd t}\frac{\rd t}{\rd x}\right]e^{-t} + \dot y \left[\frac{\rd e^{-t}}{\rd t}\ t'\right] \\
	&= \ddot y e^{-2t} - \dot y e^{-2t} = e^{-2t}\left(\ddot y - \dot y\right).
\end{align*}
Подставляем в изначальное уравнение:
\begin{align*}
\cancel{e^{2t}e^{-2t}}(\ddot y - \dot y) - 4\cancel{e^te^{-t}}\dot y + 6y &= 0; \\
\ddot y(t) - 5\dot y(t) + 6y(t) &= 0.
\end{align*}
Дальше решаем это однородное уравнение:
\[
	\lambda^2 - 5\lambda + 6 = 0;~~~~ \lambda_{1,2} = 2,3.
\]
И значит 
\begin{align*}
	y(t) &= C_1e^{2t} + C_2e^{3t}.
\intertext{Поскольку из замены $t = \ln x$, и $k\cdot\ln x = \ln x^k$, при возвращении к переменной $x$ получим}
	y(x) &= C_1x^2 + C_2x^3.
\end{align*}

\paragraph{Филиппов 575}
\[
y'' - 2y' + y = \frac{e^x}{x}.
\]
Фаза 1.
Характеристическое уравнение $\lambda^2-2\lambda+1=0$ имеет единственный кратный корень $\lambda = 1$; соответственно, ФСР -- $\lbrace e^x, xe^x\rbrace$, и общее решение однородного уравнения 
\[
	y(x) = C_1e^x + C_2 xe^x.
\]

Фаза 2.
Ищем частное решение $y_1(x)$ неоднородного уравнения в виде $y_1(x) = u_1(x)e^x + u_2(x)xe^x$. Составляем систему
\[
\begin{cases}
	u_1'e^x + u_2'xe^x &= 0, \\
	u_1'e^x + u_2'\left(e^x+xe^x\right) &= \frac{e^x}{x}.
\end{cases}
\]
В матричном виде это выглядит как
\[
\begin{pmatrix}
e^x & xe^x \\
e^x & (x+1)e^x
\end{pmatrix} \vec u' = \begin{pmatrix}
0 & e^x/x
\end{pmatrix}.
\]
Вычисляем определитель Вронского: $\Delta = e^{2x}(x+1) - xe^{2x} = e^{2x} > 0$. Как и следовало ожидать, вронскиан ФСР не равен нулю.

По методу Кронекера, вычисляем определители для искомых функций:
\begin{align*}
	\Delta_{u_1'} &= \begin{vmatrix}
	0 & xe^x \\
	e^x/x & (x+1)e^x
	\end{vmatrix} = -e^{2x}; \\
	\Delta_{u_2'} &= \begin{vmatrix}
	e^x & 0 \\
	e^x & e^x/x
	\end{vmatrix} = e^{2x}/x.
\end{align*}
Тогда
\begin{align*}
	u_1' &= \frac{\Delta_{u_1'}}{\Delta} = -1 \Rightarrow u_1 = -x + \tilde{C_1}, \\
	u_2' &= \frac{\Delta_{u_2'}}{\Delta} = \frac1x \Rightarrow u_2 = \ln|x| +\tilde{C_2}.
\end{align*}

Записываем частное решение
\begin{align*}
	y_1(x) &= \tilde{C_1}e^x \underbrace{- xe^x + \tilde{C_2}xe^x} + x\ln|x|e^x \\
		&= \tilde{C_1}e^x + \underbrace{\left(\tilde{C_2}-1\right)}_{\tilde{C_3}}xe^x + x\ln|x|e^x;
\shortintertext{и тогда общее решение неоднородного уравнения:}
	y(x) &= \hat{C_1}e^x + \hat{C_2}xe^x+x\ln|x|e^x \\
	&= e^x\left(\hat{C_1} + \hat{C_2}x + x\ln|x|\right).
\end{align*}
\begin{thebibliography}{9}
	\bibitem{Filippov}
	А.Ф. Филиппов. Сборник задач по дифференциальным уравнениям.
	\url{http://kvm.gubkin.ru/pub/uok/FilippovDU.pdf}
\end{thebibliography}
\end{document}