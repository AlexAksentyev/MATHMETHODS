\documentclass[12pt]{report}
\usepackage{../header}


\title{Тема 4. Математические модели электродинамики.}

\begin{document}
	\maketitle

\begin{tcolorbox}
	\textbf{УЧЕБНИК ДНЯ:}
	А.Н. Тихонов, А.А Самарский. Уравнения математической физики.
	\url{http://old.pskgu.ru/ebooks/tihonov.html}
\end{tcolorbox}

Эта тема посвящена поиску распределения потенциала в пространстве, то есть решению краевой задачи для уравнения Лапласа. (То есть в учебнике Тихонова нас интересует глава 4.)

\paragraph{Типы краевых задач}
Любая краевая задача -- это решение дифференциального уравнения в частных производных (урмата) $Lu = f(x)$ в некоторой области $\Omega$ (внутренняя задача), или вне её (внешняя задача), при заданных на границе области (обозначается как $\partial\Omega$) \emph{граничных уcловиях}:
\begin{enumerate}[(1)]
	\item[\textbf{Дирихле}] граничные условия налагаются на значения \emph{самой} функции: 
	\[
		u\vert_{\partial\Omega}=g(x).
	\]
	\item[\textbf{Неймана}] граничные условия налагаются на \emph{производную} функции: 
	\[
		\left.\frac{\partial u}{\partial\vec n}\right\vert_{\partial\Omega} = g(x),
	\]
	$\vec n$ -- нормаль к $\partial\Omega$. То есть, \emph{физический смысл}: задан поток поля $-\nabla u$ через границу.
	\item[\textbf{Ньютона}] смесь первых двух: 
	\[
		\left.\left(a\cdot u + b\cdot \frac{\partial u}{\partial\vec n}\right)\right\vert_{\partial\Omega} = g(x).
	\]
\end{enumerate}

\section{Потенциалы простого и двойного слоя}
Пусть $x\in\mathbb{R}^3$.
Функция $\Phi(x) = |x|^{-1}$ является \emph{фундаментальным решением} уравнения Лапласа (в пространстве).~\cite[стр.~282]{Tikhonov}

Пусть нам задана некоторая функция $\rho(x)$.
\paragraph{Объёмный потенциал}
$\rho: \Omega \mapsto \mathbb{R}$.
\begin{equation}
	u(x) = \int_\Omega \rho(y)\Phi(x-y)\rd V(y)
\end{equation}
\paragraph{Потенциал простого слоя}%~\cite[стр.~346, ур-е (26)]{Tikhonov}
$\rho: \partial\Omega \mapsto \mathbb{R}$
\begin{equation}
	\bar{u}(x) = \int_{\partial\Omega}\rho(y)\Phi(x-y)\rd S(y).
\end{equation}
См.~\cite[стр.~346, ур-е (26)]{Tikhonov}, и~\cite[ур-е (5.1)]{Stanford:potential-theory}. \\
Служит решением задачи Неймана.
\paragraph{Потенциал двойного слоя}%~\cite[стр.~348, ур-е (28)]{Tikhonov}
$\rho : \partial\Omega \mapsto \mathbb{R}$.
\begin{equation}
	\bar{\bar{u}}(x) = -\int_{\partial\Omega} \rho(x)\frac{\partial}{\partial\vec n}\Phi(x-y)\rd S(y).
\end{equation}
См.~\cite[стр.~348, ур-е (28)]{Tikhonov} и~\cite[ур-е (5.2)]{Stanford:potential-theory}. \\
Служит решением задачи Дирихле.

Собственно, решение задач Дирихле и Неймана сводится к отысканию функции $\rho(x)$, такой, что на границе $\partial\Omega$ она примыкает к граничному условию.~\cite[стр.~15]{Stanford:potential-theory} 

\begin{rmk}[Откуда уравнения Фредгольма на стр. 15?]
	Дело в том, что для потенциалов верны предельные уравнения~\cite[стр.~353, ур-е (39)]{Tikhonov}:
	\begin{align*}
		\lim_{x\in\Omega \to x_0}\bar{\bar{u}} &= \bar{\bar{u}}(x_0) + \pi\rho(x_0),\\
		\lim_{x\in\Omega^c \to x_0}\bar{\bar{u}} &= \bar{\bar{u}}(x_0) - \pi\rho(x_0).
	\end{align*}
	и~\cite[стр.~359, ур-е (48)]{Tikhonov}
	\begin{align*}
		\lim_{x\in\Omega \to x_0}\frac{\partial\bar{u}}{\partial\vec n} &= \frac{\partial\bar{u}}{\partial\vec n}(x_0) + 2\pi\rho(x_0), \\
		\lim_{x\in\Omega^c \to x_0}\frac{\partial\bar{u}}{\partial\vec n} &= \frac{\partial\bar{u}}{\partial\vec n}(x_0) - 2\pi\rho(x_0).
	\end{align*}
	Где $x_0\in\partial\Omega$.
\end{rmk}

\section{Полиномы Лежандра}
Фундаментальное решение уравнения Лапласа $\Phi(x)$ можно разлоижть в степенной ряд. Для этого его сначала приводят в форму:~\cite[стр.~672]{Tikhonov}
\begin{align*}
\Phi(r,r_0) &=  \frac{1}{\sqrt{r_0^2 + r^2 -2rr_0\cos\theta}} \tag{теорема косинусов}\\
&= \frac1r\frac{1}{\sqrt{1 + \rho^2 -2\rho x}} \\
&= \frac1r\Psi(\rho, x),
\end{align*}
где $\rho = r_0/r < 1$, $x = \cos\theta\in [-1, 1]$, а $\Psi(\rho, x)$ называется \emph{производящей функцией} полиномов Лежандра $P_n(x)$:
\[
\Psi(\rho, x) = \sum_{n=0}^\infty P_n(x)\rho^n.
\]

Полиномы Лежандра:
\begin{itemize}
	\item являются решениями дифференциального \emph{уравнения Лежандра}
	\[
	\frac{\rd}{\rd x}\left[(1-x^2)\frac{\rd y}{\rd x}\right] + \lambda y = 0
	\]
	(физический смысл имеют решения только при $\lambda=n(n+1)$, $n\in\mathbb{N}$);
	\item связаны рекуррентным соотношением
	\begin{align*}
	0 &= (n+1)P_{n+1}(x) - (2n+1)xP_n(x) + nP_{n-1}(x), \\
	P_n(1) &= 1, P_n(-x) = (-1)^nP_n(x), \\
	P_0(x) &= 1, P_1(x) = x.
	\end{align*}
	\item ортогональны на отрезке $x \in [-1, 1]$:
	\[
	\int_{-1}^{+1} P_n(x)\cdot P_m(x)\rd x = \frac{2}{2n+1}\delta_{nm}
	\]
	(обратите внимание на параграф 5 в~\cite{Tikhonov:Legandre}: норма полиномов Лежандра. Оттуда появился коэффициент $2/(2n+1)$.)
\end{itemize}
\paragraph{Сейчас вылетит птичка}
Имеем уравнение Лапласа в сферических координатах:
\[
\Delta u = 0, u = u(r,\theta,\phi),
\]
и пусть в задаче присутствует симметрия относительно $\phi$: $u = u(r,\theta)$.

Уравнение станет
\[
\frac{1}{r^2}\frac{\partial}{\partial r}\left[r^2\frac{\partial u}{\partial r}\right] + \frac{1}{r^2\sin\theta}\frac{\partial}{\partial\theta}\left[\sin\theta\frac{\partial u}{\partial\theta}\right] = 0
\]
Разделяем переменные, записываем задачу Штурма-Лиувилля:
\begin{align*}
	u &= R(r)\Theta(\theta), \\
	\frac{-\frac{\rd}{\rd r}\left[r^2\frac{\rd R}{\rd r}\right]}{R} &= \frac{\frac{1}{\sin\theta}\frac{\rd}{\rd\theta}\left[\sin\theta\frac{\rd\Theta}{\rd\theta}\right]}{\Theta} = -\lambda, \tag{Задача Ш-Л}\\
	\frac{1}{\sin\theta}\frac{\rd}{\rd\theta}\left[\sin\theta\frac{\rd\Theta}{\rd\theta} \right]+ \lambda\Theta &= 0, 
	\intertext{заменим $x = \cos\theta$, и ВНЕЗАПНО}
	\frac{\rd}{\rd x}\left[(1-x^2)\frac{\rd\Theta}{\rd x}\right] + \lambda\Theta = 0. \tag{уравнение Лежандра}
\end{align*}
 А это значит, что
 \[
 \boxed{\Theta_n(\theta) = P_n(\cos\theta).}
 \]
 
 \subsection{Присоединённые функции Лежандра}
 Присоединённые функции Лежандра появляются когда мы не ограничиваем искомую функцию:
 \[
 u = u(r,\theta,\phi) = R(r) Y(\theta,\phi).
 \]
 
 При разделении переменных, получим задачу Ш-Л для функции $Y(\theta,\phi)$ в виде
 \[
 \Delta_{\theta,\phi} Y + \lambda Y = 0, Y(\theta, \phi+2\pi) = Y(\theta, \phi).
 \]
 Снова разделяем переменные
 \[
 Y(\theta, \phi) = \Theta(\theta)\Phi(\phi);
 \]
 обозначив $x = \cos\theta$ придём к уравнению
 \begin{equation}\label{eq:Legandre-plus}
 \frac{\rd}{\rd x}\left[(1-x^2)\frac{\rd\Theta}{\rd x}\right] + \left(\lambda - \frac{m^2}{1-x^2}\right)\Theta = 0.
 \end{equation}
Здесь
\begin{enumerate}[(1)]
	\item $\lambda = n(n+1)$, $n\in\mathbb{N}$, чтобы $|\Theta| < \infty$;
	\item $m\le n$, чтобы $P_n^{(m)}(x) ~\cancel{\equiv}~ 0$.
\end{enumerate}

Решением уравнения~\ref{eq:Legandre-plus} является \emph{присоединённая функция Лежандра}
\[
P_n^{(m)} = (1-x^2)^{\sfrac m2}\frac{\rd^m}{\rd x^2}P_n(x).
\]

Собираем решение:
\begin{align*}
Y_n^{(m)}(\theta, \phi) &= P_n^{(m)} (\cos\theta) \cdot \Phi_m(\phi) \\
	&= P_n^{(m)}(\cos\theta)\cdot 
	\begin{cases}
	\cos m\phi, & m \le 0, \\
	\sin m\phi, & m \ge 1.
	\end{cases}
\end{align*}
\begin{rmk}[Формальное разделение переменных]
	В последнем равенстве мы формально разделили $\sin, \cos$ для значений $m$ разных знаков чтобы не писать $\sum(\sin + \cos)$.
\end{rmk}
\begin{rmk}[Фундаментальная сферическая функция]
	Именно так называется функция $Y_n^{(m)}$.
\end{rmk}
\begin{rmk}[Сферическая гармоника]
	\begin{align*}
	Y_n &= \sum_{m=-n}^n C_{n,m}Y_n^{(m)}(\theta, \phi) \\
	&= \sum_{m=0}^n\left(A_{n,m}\cos m\phi + B_{n,m}\sin m\phi\right) P_n^{(m)}(\cos\theta).
	\end{align*}
\end{rmk}

\section{Задачи по поиску поля по распределению заряда}
На сколько я понимаю, это задачи на применение \textbf{теоремы Гаусса} и \textbf{закона полного тока}. 
Теорема Гаусса
\[
\int_S \vec E\cdot \rd\vec s = \frac{Q}{\epsilon_0}
\]
позволяет вычислять электрическое поле заряда  $Q$, распределённого в пространстве с плотностью $\rho$.

Альтернативный для магнитного поля закон полного тока
\[
\oint_L\vec B\rd\vec\ell = \mu_0 I
\]
позволяет вычислять магнитное поле, создаваемое током $I$, распределённым в пространстве с некоторой плотностью $j$.

\section{Электростатическое поле внутри бесконечной призмы}
Задача состоит в следующем: есть четыре бесконечных (в длину) проводящих электрода, каждый под каким-то своим потенциалом. Надо найти распределение потенциала внутри призмы.
Эта задача решается простым разделением переменных. Нужно только отметить, что, поскольку призма \textit{бесконечно длинная}, потенциал не зависит от продольной координаты. То есть, остаётся просто решить задачу Дирихле для двуменрого уравнения Лапласа:
\begin{equation*}
	\begin{cases}
		\Delta u &= 0, \\
		u(x=0) &= u_1, \\
		u(x=a) &= u_2, \\
		u(y=0) &= u_3, \\
		u(y=b) &= u_4.
	\end{cases}
\end{equation*}

\textbf{Важно!} при решении уравнения методом разделения переменных предполагается, что граничные условия однородные (нулевые). Это нужно потому (на сколько я могу судить), что при решении задачи Штурма-Лиувилля, если условия неоднородные, невозможно разделить аргумент функции-решения, и коэффициент перед ней. Если же условие нулевое, то всё просто. Например, при решении данной задачи, получим уравнение
\[
X'' + \lambda^2 X = 0,
\]
с решением $X_n(x) = A_n \cos(\sqrt{\lambda}\cdot x) + B_n \sin(\sqrt{\lambda}\cdot x)$.
Для примера, возьмём $X(0) = u_1$. Тогда:
\[
u_1 = A_n\cos(\sqrt{\lambda}\cdot 0) = A_n.
\]
Пока всё относительно хорошо, но теперь для второго граничного условия для $X(x)$:
\[
	u_2 = u_1\cos(\sqrt{\lambda}\cdot a) + B_n\sin(\sqrt{\lambda}\cdot a).
\]
И что делать? А если бы условия были нулевые, всё было бы решаемо.

Поэтому, чтобы сделать граничные условия однородными, надо воспользоваться линейностью уравнения Лапласа, и разделить задачу на четыре~\cite{Laplace-eq-rectangle}, в каждой из которых достаточное количество нулевых граничных условий. Ненулевое граничное условие нужно будет разложить в ряд Фурье, и таким образом отыскать значение недостающего коэффициента, который не определяется из нулевых условий.


\begin{thebibliography}{0}
	\bibitem{Tikhonov}
	А.Н. Тихонов, А.А. Самарский.
	Уравнения математической физики.
	\url{http://ijevanlib.ysu.am/wp-content/uploads/2018/01/converted_file_81e63622.pdf}
	\bibitem{Stanford:potential-theory}
	Potential theory.
	\url{https://web.stanford.edu/class/math220b/handouts/potential.pdf}
	\bibitem{Problems:Gauss+Stokes}
	Теорема Гаусса  её применение к вычислению электрических полей простейших распределений плотности заряда.
	\url{http://phys.spbu.ru/content/File/Library/studentlectures/Krylov/Gos_Ekzam-13-14-1.pdf}
	\bibitem{Problem-solutions}
	А.Н. Паршаков. Принципы и практика решения задач по общей физике.
	\url{http://pstu.ru/files/file/oksana/2011/fakultety_i_kafedry/fpmm/prikladnaya_fizika/informacionnye_resursy/principy_i_praktika_resheniya_zadach_po_obschey_fizike__chast_2__elektromagnetizm.pdf}
	\bibitem{Kuptsov}
	А.М. Купцов. Теоретические основы электротехники. Решения типовыз задач.
	\url{http://window.edu.ru/resource/045/76045/files/emp.pdf}
	\bibitem{Laplace-eq-rectangle}
	Anthony Peirce. Lecture 24: Laplace's equation.
	\url{https://www.math.ubc.ca/~peirce/M257_316_2012_Lecture_24.pdf}
	\bibitem{Tikhonov:Legandre}
	А.Н. Тихонов, А.А. Самарский.
	Уравнения математический физики. 
	Дополнение  II Специальные функции. 
	Часть II Сферические функции.
	\url{http://old.pskgu.ru/ebooks/ts/ts_dop2_2_1.pdf}
\end{thebibliography}
\end{document}