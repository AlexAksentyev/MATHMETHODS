\documentclass{report}

\usepackage{../../header}


\title{Тема 4. Задача 12}

\begin{document}
	\maketitle
	
\paragraph{Условие задачи: }
Решить уравнение Пуассона \textbf{в кольце}
\begin{align*}
	\Delta u &= 8r^{-5}\sin\phi, \\
	u(1,\phi) &= \sin\phi, \\
	u'_r(2,\phi) &= -12\sin\phi.
\end{align*}

Решение ищем в виде суммы общего решения \emph{однородного уравнения} с \emph{неоднородными граничными условиями},
и частного решения \emph{неоднородного уравнения} с \emph{однородными граничными условиями}.

\section{Отыскание решения однородного уравнения}
...
\section{Отыскание частного решения}
Будем решать частное решение уравнения в виде функции в правой части: 
\[
u_1(r,\phi) = W(r)\sin\phi.
\]

Поставив $u_1$ в исходное уравнение и сократим $\sin\phi$:
\[
\cancel{\sin\phi}\frac1r[W' + rW''] - \frac{W}{r^2}\cancel{\sin\phi} = 8r^{-5}\cancel{\sin\phi}.
\]

В итоге, надо решить уравнение
\begin{equation}\label{eq:Difur}
	r^2W'' + rW' - W = 8r^{-3}.
\end{equation}

Введём замену $r = e^t$. Напоминаю, что:
\begin{align*}
	\frac{\rd t}{\rd r} &= \frac1r, \\
	\frac{\rd W}{\rd r} &= \frac{\rd W}{r\rd t}, \\
	\frac{\rd^2 W}{\rd r^2} &= \frac{\rd^2 W}{r^2\rd t^2}.
\end{align*}

\href{http://www.studentlibrary.ru/doc/bauman_0018-SCN0006/001.html}{Здесь} подобное уравнение решается.
Каким образом они свели (5.2) к (5.3) через эту замену я не очень понял. У меня $W'$ не уходит.

Но в целом, понятно, что получаем диффур, который надо решить.


\end{document}