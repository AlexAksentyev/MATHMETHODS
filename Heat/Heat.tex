\documentclass[12pt]{report}
\usepackage{../header}


\title{Тема 5. Математические модели теплопроводности диффузии}

\begin{document}
\maketitle

\begin{tcolorbox}
  \textbf{УЧЕБНИК ДНЯ:}
  А.Н. Тихонов, А.А Самарский. Уравнения математической физики.
  Глава III: Уравнения параболического типа.
  \url{http://old.pskgu.ru/ebooks/tihonov.html}
\end{tcolorbox}

\paragraph{Уравнение диффузии}
\begin{equation*}
  \frac{\partial}{\partial t}\rho = \nabla\cdot \left[D\nabla \rho\right] + f
\end{equation*}
Здесь 
\begin{itemize}
	\item  $\rho(\vec r,t)$ -- плотность диффундирующего вещества,
	\item $D(\rho, \vec	r)$ -- коэффициент диффузии,
	\item $f(\vec r,t)$ описывает источники вещества.
\end{itemize}

\paragraph{Уравнение теплопроводности}
\begin{equation*}
  c\rho\frac{\partial}{\partial t} T = \nabla\cdot\left[D \nabla T\right] + f
\end{equation*}
Где 
\begin{itemize}
	\item $T(\vec r, t)$ -- температура,
	\item $D(\vec r)$ -- коэффициент теплопроводности,
	\item $f(\vec r,t)$ описывает источники тепла,
	\item $c$ -- теплоёмкость,
	\item $\rho$ -- плотность вещества.
\end{itemize}

Собственно, при записи в общем виде -- это одно и то же уравнение.

\begin{rmk}
	Когда в задаче упоминается слово \emph{стационарный}, это значит, что производная по времени равна нулю:
	\[
	 	\frac{\partial}{\partial t} u \equiv 0.
	\]
	Если при этом коэффициент $D=\const$, получаем уравнение Пуассона:
	\[
	\Delta u = f.
	\]
\end{rmk}

Чтобы решить уравнение параболического типа, нам нужны:
\begin{enumerate}
	\item граничные условия (Дирихле, Неймана, Робена);
	\item начальные условия (задача Коши).
\end{enumerate}

В задачах из учебного плана нет неоднородных уравнений. Уравнения решаем методом разделения переменных:~\cite{diffusion-equation}
\begin{enumerate}
	\item находим функцию, удовлетворяющую граничным условиям;
	\item раскладываем начальные условия в ряд Фурье,~\cite{Fourier-series} и отыскиваем коэффициенты.
\end{enumerate}


\section{Примеры}
\subsection{Задача на отрезке}
Решить уравнение теплопроводности
\begin{align}
	4\frac{\partial^2T}{\partial z^2} &= \frac{\partial T}{\partial t}, \label{eq:heat-main}\\
	T(0, z) &= \sin 2\pi z - \sin 4\pi z, \label{eq:heat-initial}\\
	T(t,0) &= T(t,2) =0.\label{eq:heat-boundary}
\end{align}

Всё стандартно, как описано в~\cite{diffusion-equation}. Решаем метдом разделения переменных, полагая
$T = Z(z)\cdot W(t)$. Решение уравнения сводится к разложению начальных условиий в ряд Фурье.

\begin{rmk}{Ортогональность собственных функций}
	Обратите внимание, что в начальных условиях~\eqref{eq:heat-initial}
	задачи~\eqref{eq:heat-main}--~\eqref{eq:heat-boundary} стоят собственные функции
	задачи Штурма-Лиувилля для функции $Z(z)$.
	Собственные функции ортогональны~\cite[стр.~17]{UMF-problems-solutions}, поэтому 
	в решении $T = \sum_{n=0}^{\infty} Z_n(z)W_n(t)$ останутся только два члена: при 
	$Z_n(z) = \sin2\pi z$ и $Z_n(z) = \sin4\pi z$.
\end{rmk}

\subsection{Задача в круге}
Решить уравнение
\begin{align}
2\Delta T &= \frac{\partial T}{\partial t}, \label{eq:circle}\\
T(0, r) &= 9 - r^2,\\
T(t,3) &=0,
\end{align}
в круге (радиусом $r_0 = 3$).

В этой задаче мы впервые встречаемся с функциями Бесселя.

Перейдём в полярную систему коордиинат $(r,\phi)$. Судя по начальным условииям, задача обладает симметрией относительно угла $\phi$; таким образом, решение будем искать в виде 
\[
T(t,r) = R(r)W(t).
\]

Поскольку зависимости от $\phi$ нет, оператор Лапласа запишется как 
\[
\Delta T = \frac{1}{r}\frac{\rd }{\rd r}\left(r \frac{\rd T}{\rd r}\right),
\]
и задача Ш-Л для уравнения~\eqref{eq:circle} приобретёт вид
\begin{align*}
	\frac{r^{-1}\left[R' + rR''\right]}{R} = \frac12\frac{W'}{W} = \lambda = -a^2 = \const.
\end{align*}

Я принял $\lambda = -a^2$ для удобства, зная, что будет дальше. Запиишем уравнение для $R$:
\begin{equation}\label{eq:almost-bessel}
R'' + r^{-1}R' + a^2R = 0.
\end{equation}

Уравнение Бесселя выглядит вот так:~\cite[стр.~625]{TS:special-funcs}
\begin{equation}\label{eq:bessel}
	x^2y'' + xy' + (x^2 - \nu^2)y=0.
\end{equation}

Обратите внимание на ссылку~\cite{bessel-funcs}; и в особенности на \textbf{пункт 3} в \emph{Некоторые дифференциальные уравнения, приводимые к уравнению Бесселя}. Чтобы получить уравнение из пункта \textbf{3}, домножим уравнение~\eqref{eq:almost-bessel} на $r^2$, и добавим к нему член $-\nu^2R \equiv 0$, при $\nu=0$:
\begin{equation}\label{eq:almost-bessel-1}
	r^2R'' + rR' + (a^2r^2-\nu^2)R = 0.
\end{equation}

Решением уравнения~\eqref{eq:almost-bessel-1} является $R(r) = C_1J_\nu(ar) + C_2Y_\nu(ar)$, где $J_\nu, Y_\nu$ -- функции Бесселя первого и второго рода (порядка $\nu$) соответственно.

\begin{rmk}
	Обратите внимение, что $0\in\mathbb{Z}$ (целое число); поэтому решение записывается в виде суммы функциий Бесселя первого и второго родов. Если бы $\nu\notin\mathbb{Z}$, было бы
	\[
	R(r) = C_1J_\nu(ar) + C_2J_{-\nu}(ar).
	\]
\end{rmk}

\begin{rmk}
	Кому интересно, сведение уравнения~\eqref{eq:almost-bessel-1} к~\eqref{eq:bessel} производится путём замены
	\begin{align*}
		x &= ar, \\
		\rd r &= a^{-1}\rd x, \\
		\frac{\rd}{\rd r} &= a\frac{\rd}{\rd x},\\
		\frac{\rd^2}{\rd r^2}&= a^2\frac{\rd^2}{\rd x^2}.
	\end{align*}
\end{rmk}

Решение уравнения для $W$ даёт
\[
W(t) = C_3e^{-2a^2t}.
\]

Собираем решение для $T$:
\[
T(t,r) = \left[AJ_0(ar) + BY_0(ar)\right]e^{-2a^2t}.
\]

Теперь определим коэффициенты.
\begin{enumerate}
	\item Так как $Y_0(ar)$ не ограничена при стремлении к нулю, это значит $B=0$ (аргумент от нефизичности решения).
	\item Поскольку $T(t,3) = R(3)W(t) = 0$, и мы \emph{требуем} $W(t) \not\equiv 0$,
	значит $R(3) = 0$. Это условие позволяет нам определить $a$: потребуем, чтобы $3a = \mu^1_0$, где $\mu^1_0$ это первый корень $J_0$: $J_0(\mu^1_0) = 0$; $a = \mu^1_0/3$. 
\end{enumerate}
 

\begin{thebibliography}{0}
	\bibitem{diffusion-equation}
	The diffusion equation.
	\url{https://www.uni-muenster.de/imperia/md/content/physik_tp/lectures/ws2016-2017/num_methods_i/heat.pdf}
	\bibitem{Fourier-series}
	Fourier Series.
	\url{http://mathworld.wolfram.com/FourierSeries.html}
	\bibitem{bessel-funcs}
	Уравнение Бесселя.
	\url{http://www.math24.ru/уравнение-бесселя.html}
	\bibitem{TS:special-funcs}
	А.Н. Тихонов, А.А Самарский. Уравнения математической физики.
	Дополнение II: Специальные функции.
	\url{http://old.pskgu.ru/ebooks/ts/ts_dop2_0.pdf}
	\bibitem{UMF-problems-solutions}
	С.В. Ревина, Л.И. Сазонов, О.А. Цывенкова.
	Уравнения математической физики. Задачи и решения.
	\url{http://www.mmcs.sfedu.ru/jdownload/finish/16-kafedra-vychislitelnoj-matematiki-i-matematicheskoj-fiziki/1419-uravneniya-matematicheskoj-fiziki-zadachi-i-resheniya-s-v-revina-l-i-sazonov-o-a-tsyvenkova}
\end{thebibliography}
      
\end{document}