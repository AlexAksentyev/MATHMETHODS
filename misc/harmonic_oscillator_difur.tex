\documentclass{report}

\usepackage{../header}


\title{Решениие уравнения гармонического осцииллятора}
\begin{document}
	\maketitle
	
Однородное уравнение гармонического осциллятора:
\begin{equation}\label{eq:oscillator}
x'' + \omega^2x = 0.
\end{equation}

Это линейное однородное уравнение второго порядка. Запишем характеристическое уравнение:
\[
\lambda^2 + \omega^2 = 0.
\]

Корни характеристического уравнения: $\lambda_{1,2} = \pm i\omega$, вследствие чего, решение
уравнения~\eqref{eq:oscillator} запишется в виде
\begin{align*}
x(t) &= C_1 e^{i\omega t} + C_2 e^{-i\omega t}\\
&= C_1(\cos\omega t + i\sin\omega t) + C_2(\cos\omega t - i\sin\omega t) \\ 
&= (C_1+C_2)\cos\omega t + i(C_1-C_2)\sin\omega t.
\end{align*}
Здесь, коэффициенты $C_1,C_2\in\mathbb{C}$ -- комплексные числа. При этом (!) если мы требуем, чтобы $x(t)\in\mathbb{R}$,  то они комплексно-сопряжённые:
\begin{align*}
C_1 &= a + ib, \\
C_2 &= a - ib.
\end{align*}

Сооответственно, 
\begin{align*}
	C_1 + C_2 &= 2a \equiv A\\
	C_1 - C_2 &= 2ib \equiv iB,
\end{align*} 
и тогда
\begin{align}
x(t) &= A\cos\omega t - B\sin\omega t \notag\\
&= A\cos\omega t + B_1\sin\omega t.
\end{align}

\begin{rmk}[Доказательство комплексной сопряжённости $C_1$ и $C_2$]
	Если $x(t)\in\mathbb{R}$, то $x^*(t) = x(t)$ (комплексно-сопряжённое с собой). Тогда:
	\begin{align*}
		x^*(t) &= C_1^*(e^{i\omega t})^* + C_2^*(e^{-i\omega t})^* \\
		&= C_1^* e^{-i\omega t} + C_2^* e^{i\omega t} \\
		&= C_2 e^{-i\omega t} + C_1 e^{i\omega t}.
	\end{align*}
	Поскольку $e^{i\omega t}\perp e^{-i\omega t}$:
	$C_1^* = C_2$.
\end{rmk}
\end{document}